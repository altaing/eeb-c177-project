%preamble

\documentclass[letterpaper]{article}
\usepackage[utf8]{inputenc}
\usepackage[margin=1in]{geometry}
\usepackage{graphicx}
\graphicspath{{images/}}
\usepackage{color}
\usepackage{listings}

\title{Outline}
\author{Alton Taing}
\date{\today}

\begin{document}
\maketitle
\section*{Abstract}
Lorem ipsum dolor sit amet, consectetur adipiscing elit, sed do eiusmod tempor incididunt ut labore et dolore magna aliqua. Ut enim ad minim veniam, quis nostrud exercitation ullamco laboris nisi ut aliquip ex ea commodo consequat. Duis aute irure dolor in reprehenderit in voluptate velit esse cillum dolore eu fugiat nulla pariatur. Excepteur sint occaecat cupidatat non proident, sunt in culpa qui officia deserunt mollit anim id est laborum.

\newpage
\tableofcontents
\listoffigures
\newpage

\section{Introduction}
The rationale behind analyzing this dataset is to determine if there are any trends between characteristics of patients with differing levels of heart disease. Knowing these trends will perhaps enable us to better understand and treat those affected and make predictions on the direction their health may be heading. 
*include papers about heart disease?*

\section{Methods}

\begin{lstlisting}[language=Python]

# MAIN FUNCTION
op = '' # defining variable op
clmns = [] # defining list clmns
filename='' # defining variable filename
def shellfunction(filename, clmns, op): # defining function with 3 inputs
    import pandas as pd # importing pandas
    import seaborn as sns # importing seaborn for plotting 
    import re #importing regex
    print('Input .csv file you wish to analyze')
    filename = str(input()) # allows user to input filename
    assert check(filename) == True, 'This function will only work with .csv files\nPlease input a .csv file.'
    print('\n') # the assert statement will ensure we are working with a .csv file
    cl = [] # making an empty list
    clmns = cl # making the empty list equal to clmns
    print('age = 0, sex = 1\nchest pain = 2, resting blood pressure = 3\ncholesterol = 4, fasting blood sugar = 5\nresting ECG = 6, max heart rate = 7\nexercise induced angina = 8, ST depression induced by excercise = 9\nslope of the peak exercise ST segment = 10\nnumber of major vessels colored by flourosopy = 11\nthal = 12, target = 13')
    print('\n') # providing a key
    for i in range(0, 2): # making a for loop where the user can input what elements they want to use
        print("Enter number corresponding to desired element to be plotted")
        item = int(input()) 
        cl.append(item) # appending user input to the empty list made earlier
    x, y = cl # setting up assert statement
    assert x != y, 'Please input different values' # using assert statement to ensure differnt values are used
    print('\n')
    df = pd.read_csv(filename, usecols = clmns) # importing file and designating columns to use from pandas
    op = str(input("What operation would you like to run on this data?:\nAverage(A), Maximum(MX), Minimum(MN), Standard Deviation(STD)\nType A, MX, MN, or STD: "))
    print('\n') # providing instructions for the user
    if op == "A" or op == "a": # set up if, elif, else statements to take on use inputs
        avg = df.mean(axis=0) # average function
        v1, v2 = df
        sns.lmplot(v1, v2, data=df,fit_reg=True) #plotting using seaborn
        print(avg)
    elif op == "MX" or op == "mx":
        mx = df.max(axis=0) # max function
        v1, v2 = df
        sns.lmplot(v1, v2, data=df,fit_reg=True)
        print(mx)
    elif op == "MN" or op == "mn":
        mn = df.min(axis=0) # min function
        v1, v2 = df
        sns.lmplot(v1, v2, data=df,fit_reg=True)
        print(mn)
    elif op == "STD" or op == "std":
        stdev = df.std(axis=0) # standard deviation function
        v1, v2 = df
        sns.lmplot(v1, v2, data=df,fit_reg=True)
        print(stdev)
    else:
        print("Invalid Input")


# SUB FUNCTION
def check(file): # defining function check
    import re  # importing regex
    csv = re.compile(r'.*\.csv') # will search for input ending in .csv
    file = csv.search(file) #  returns first match
    return bool(file) # if the input ends in .csv, this function will return True
    
    
\end{lstlisting}
\section{Results}
Lorem ipsum dolor sit amet, consectetur adipiscing elit, sed do eiusmod tempor incididunt ut labore et dolore magna aliqua. Ut enim ad minim veniam, quis nostrud exercitation ullamco laboris nisi ut aliquip ex ea commodo consequat. Duis aute irure dolor in reprehenderit in voluptate velit esse cillum dolore eu fugiat nulla pariatur. Excepteur sint occaecat cupidatat non proident, sunt in culpa qui officia deserunt mollit anim id est laborum.

\section{Discussion}
Lorem ipsum dolor sit amet, consectetur adipiscing elit, sed do eiusmod tempor incididunt ut labore et dolore magna aliqua. Ut enim ad minim veniam, quis nostrud exercitation ullamco laboris nisi ut aliquip ex ea commodo consequat. Duis aute irure dolor in reprehenderit in voluptate velit esse cillum dolore eu fugiat nulla pariatur. Excepteur sint occaecat cupidatat non proident, sunt in culpa qui officia deserunt mollit anim id est laborum.

\newpage

\bibliography{References}
\bibliographystyle{plain}
\end{document}